% 2015-TS-ISN-ApprendreLaTeX.tex
% Mise à jour : 02/04/2016
% 
\documentclass[a4paper,11pt,french]{article}
\usepackage[utf8]{inputenc}    
\usepackage[T1]{fontenc}
\usepackage[francais]{babel}
\usepackage{graphicx}
\title{Compte Rendu : Bomberman}
\author{Tanguy \bsc{Thomas}, Titouan \bsc{Rannou}, Clément \bsc{Guin}}
\date{17 mai 2016}
\begin{document}
\maketitle

%& \grave{a}
\section{Cahier des Charges}
\subsection{Presentation}
Le but de ce projet est de créer un clone du jeu vidéo \textit{Bomberman}. Il s'agit d'un jeu multijoueur qui se joue à 2 joueurs. Voici les régles :
\subsubsection{Regles du jeu}
Imaginez une arène (vue de dessus) composée de blocs incassables, de blocs cassables (briques) et de chemins praticables. Le jeu se joue à 2 joueurs, ce sont tous les 2 des Bombermans. Il peut se déplacer et poser des bombes. Après quelques secondes elles explosent et détruisent les blocs cassables proches. Les régles sont simples : il ne doit en rester qu'un. Il faut utiliser toute sa ruse, récuper des items bonus et en posant plusieurs bombes afin de venir à bout de votre adversaire.

\subsubsection{Environnement}
Nous décrivons dans cette section l'arène du jeu :
La construction de l'arène se fait à l'aide des images suivantes :
\begin{itemize}
\item Des blocs \textit{incassables} %image
\includegraphics[width=1cm,angle=0]{blocs}
\item Des blocs \textit{cassables} %image
\includegraphics[width=1cm,angle=0]{briques}
\item Des  2 personnages  %les 2 images
\includegraphics[width=1cm,angle=0]{joueur1}
\includegraphics[width=1cm,angle=0]{joueur2}
\end{itemize}
Chaque joueur commence à des coins opposés de l'arène. Ils ont chacun 3 vies en début de partie. Ils en perdent une s'ils sont dans la portée de la bombe. Si un joueur n'a plus de vie il a perdu. Chaque joueur peut poser 1 piège qui immobilise le joueur adverse
\subsubsection{Ce que le joueur peut ou doit faire}
\begin{itemize}
\item Ne rien faire
\item Se déplacer dans l'arène sur les cases vides
\item Pouvoir récuper des bonus
\item Aller sur la même case qu'un joueur, une bombe ou un bonus
\item Poser des bombes
\end{itemize}
\subsubsection{Ce que le joueur ne peut pas faire}
\begin{itemize}
\item Sortir de l'arène
\item Traverser des obstacles (blocs et briques)
\item Jouer une fois mort
\item Poser plus de 3 bombes en même temps
\end{itemize}
\subsubsection{Ce que les bombes peuvent ou doivent faire}
\begin{itemize}
\item Exploser au bout de 2 secondes
\item Être traversées par les joueurs
\end{itemize}
\subsubsection{Ce que les bombes ne peuvent pas faire}
\begin{itemize}
\item Cassser des blocs (éléments incassables)
\end{itemize}
\subsubsection{Ce que l'explosion peut ou doit faire}
\begin{itemize}
\item Tuer le joueur qui a posé la bombe ou/et le joueur adversaire
\item Casser des briques (éléments cassables) correspondant à la portée du joueur
\item Faire apparaître différents bonus
\end{itemize}
\subsubsection{Bonus} 
\begin{itemize}
\item Vie
\item Augmentation de la portée de la bombe
\end{itemize}

			% {\Large } : pour écrire en plus gros 
			% \textit{} :  pour mettre en italiques
			% pour mettre en gras {\bf course}}
			% pour entourer un mot \fbox{la vie est belle}
\newpage 
\end{document}